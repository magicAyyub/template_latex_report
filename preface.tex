% --------------------------------------------------------------
%                         Introduction
% --------------------------------------------------------------

\section*{Introduction}

Le développement rapide des technologies de l'information et de la communication a profondément transformé les économies du monde entier, engendrant une révolution numérique qui touche tous les secteurs d'activité (Castells, 2010). En Afrique subsaharienne, et particulièrement en Côte d'Ivoire, cette montée en puissance du numérique se manifeste par l'essor spectaculaire du commerce en ligne, des services de paiement mobile et des plateformes numériques, qui attirent chaque jour davantage d'usagers et redessinent les contours de l'économie locale.

À Abidjan, par exemple, il n'est plus rare que les consommateurs effectuent leurs achats sur des plateformes telles que Jumia ou Glovo, règlent leurs factures via Orange Money ou Wave, et se déplacent grâce aux applications de transport comme Yango ou Indrive. Ces innovations technologiques offrent indéniablement plus de confort, d'efficacité et d'opportunités économiques, contribuant ainsi à l'inclusion financière et à la dynamisation des échanges commerciaux.

Cependant, cette révolution numérique soulève également de nombreux défis, tant pour les usagers que pour les acteurs économiques. Les arnaques en ligne, la contrefaçon de produits, les litiges liés aux services de transport, la concurrence déloyale avec les acteurs traditionnels, ou encore la collecte massive de données personnelles sans contrôle suffisant constituent des menaces réelles pour la confiance des consommateurs et pour la stabilité de l'économie nationale. À cela s'ajoute la difficulté pour l'État de percevoir une fiscalité équitable face à ces grandes plateformes, souvent étrangères et adeptes de l'optimisation fiscale (Zuboff, 2019).

Dans ce contexte complexe, le rôle de l'autorité publique devient essentiel. Elle doit à la fois encourager l'innovation numérique, source de croissance et d'emplois, tout en mettant en place un cadre de régulation capable de protéger les consommateurs, d'assurer la transparence des marchés et de garantir une concurrence loyale. Cette dualité entre stimulation de l'innovation et protection de l'intérêt général constitue le coeur de la problématique abordée dans cette recherche.

Dès lors, une interrogation centrale émerge : comment l'autorité publique ivoirienne peut-elle réguler efficacement les marchés numériques afin d'accompagner leur essor tout en protégeant les acteurs économiques et les citoyens ?

\section*{Problématique}

En Côte d'Ivoire, les marchés numériques connaissent une croissance rapide, portée par l'usage massif du téléphone mobile et des services financiers digitaux. Des plateformes comme Jumia, Glovo, Orange Money, Wave ou encore des sites de vente en ligne internationaux sont désormais intégrées dans le quotidien des consommateurs ivoiriens. À cela s'ajoutent des applications de transport numérique comme Yango et Indrive, qui transforment en profondeur les habitudes de mobilité dans les grandes villes comme Abidjan, contribuant à une urbanisation plus fluide mais aussi à de nouvelles formes de précarité pour certains travailleurs.

Toutefois, cette expansion fulgurante s'accompagne de difficultés majeures. Les arnaques en ligne, la vente de produits contrefaits, l'absence de protection juridique suffisante pour les consommateurs, l'évasion fiscale des grandes plateformes étrangères, ainsi que la collecte et l'utilisation abusive des données personnelles sont autant de défis qui interrogent la capacité de l'État à maintenir un équilibre entre innovation et protection sociale. Du côté du transport numérique, des tensions apparaissent entre les chauffeurs traditionnels (taxis, wôro-wôrô) et les nouvelles plateformes, ce qui soulève des enjeux de concurrence et d'équité sociale, parfois exacerbés par des mouvements de protestation.

Dans ce contexte, l'État ivoirien est appelé à intervenir afin d'assurer un cadre légal adapté aux réalités du numérique. Mais il se heurte à une difficulté majeure : comment mettre en place une régulation efficace de ces marchés sans freiner leur dynamisme, ni décourager l'innovation ? Ou encore, dans quelle mesure l'autorité publique ivoirienne peut-elle assurer une régulation efficace des marchés numériques tout en favorisant leur croissance économique et en préservant les droits fondamentaux des citoyens ?

\section*{Hypothèses de recherche}

Au regard de cette problématique, plusieurs hypothèses peuvent être formulées pour guider l'analyse :

\begin{enumerate}
\item \textbf{Hypothèse 1 :} La régulation efficace des marchés numériques en Côte d'Ivoire dépend avant tout de l'existence d'un cadre juridique clair, moderne et adapté aux réalités du numérique. Cela implique la mise à jour des textes législatifs et réglementaires, ainsi que le renforcement des compétences de l'Autorité de Régulation des Télécommunications/TIC de Côte d'Ivoire (ARTCI) et des institutions partenaires, afin de combler les lacunes actuelles en matière de protection des données et de concurrence.

\item \textbf{Hypothèse 2 :} Une régulation efficace nécessite une coopération étroite entre l'État, les acteurs privés et la société civile, afin de garantir un équilibre entre innovation et protection de l'intérêt général. Ce dialogue public-privé permettrait d'élaborer des politiques inclusives favorisant à la fois la compétitivité des entreprises et la sécurité juridique des usagers, en intégrant les perspectives des différentes parties prenantes.

\item \textbf{Hypothèse 3 :} Le succès de la régulation repose également sur la sensibilisation et la formation des utilisateurs du numérique. Une population mieux informée de ses droits et obligations contribue à un environnement numérique plus sûr et plus responsable, réduisant ainsi les risques d'abus et favorisant une adoption éthique des technologies.
\end{enumerate}

Ces hypothèses guideront la présente recherche, dont les objectifs sont précisés ci-après.

\section*{Objectifs de la recherche}

\textbf{Objectif général :}

Analyser les mécanismes juridiques et institutionnels permettant à l'autorité publique ivoirienne d'assurer une régulation efficace des marchés numériques, tout en favorisant leur développement économique et en préservant les droits des citoyens.

\textbf{Objectifs spécifiques :}

\begin{enumerate}
\item Identifier le cadre juridique existant en matière de régulation des marchés numériques en Côte d'Ivoire, en examinant les textes législatifs et réglementaires applicables.

\item Évaluer le rôle et les limites des institutions chargées de cette régulation, notamment l'ARTCI et les ministères concernés, à travers une analyse de leurs compétences et de leurs actions.

\item Analyser les principaux défis liés à la régulation des secteurs du commerce et des ventes en ligne (Jumia), du transport et de la livraison numérique (Glovo, Yango) et des services de paiement mobile (Orange Money, Wave), en mettant en lumière les enjeux économiques, sociaux et juridiques.

\item Proposer des pistes d'amélioration pour renforcer l'efficacité de la régulation et promouvoir un développement numérique durable et inclusif, adapté au contexte ivoirien.
\end{enumerate}

\section*{Intérêt du sujet}

Ce sujet présente un intérêt à la fois pratique et théorique, s'inscrivant dans le débat contemporain sur la gouvernance des technologies numériques en Afrique.

\subsection*{Intérêt pratique}

\begin{itemize}
\item L'étude permet de comprendre les mécanismes de régulation mis en place par l'État ivoirien face à l'essor rapide des marchés numériques, offrant une analyse critique des politiques publiques en vigueur.
\item En analysant des cas concrets comme Jumia, Glovo, Yango, Orange Money ou Wave, elle met en lumière les difficultés rencontrées dans la protection des consommateurs, la fiscalité des plateformes étrangères et la concurrence équitable avec les acteurs traditionnels.
\item Elle apporte ainsi des propositions concrètes pour une meilleure régulation, tout en soutenant le développement du numérique comme levier de croissance économique et d'inclusion sociale.
\end{itemize}

\subsection*{Intérêt théorique}

\begin{itemize}
\item Sur le plan scientifique, le sujet contribue à enrichir la réflexion en droit public économique, particulièrement dans le champ encore peu exploré de la régulation des marchés numériques en Afrique francophone.
\item Il permet d'analyser comment les autorités publiques adaptent leurs outils juridiques et administratifs face à un secteur en constante mutation, dominé par les innovations technologiques et les acteurs privés internationaux.
\item Cette recherche s'inscrit dans les travaux sur la "gouvernance algorithmique" et la régulation des plateformes numériques, apportant une perspective africaine originale à ces débats globaux.
\end{itemize}

\section*{Méthodologie de la recherche}

La présente étude repose sur une approche essentiellement juridique et analytique, complétée par une observation du contexte socio-économique ivoirien. Elle vise à comprendre comment l'autorité publique parvient ou tente de réguler les différents marchés numériques en Côte d'Ivoire, en combinant analyse documentaire et étude de cas.

\subsection*{Type de recherche}

La recherche est avant tout théorique, car elle s'appuie sur l'analyse du cadre juridique existant (textes de loi, règlements, politiques publiques, etc.) et sur les écrits doctrinaux en matière de régulation économique et de droit du numérique (Lessig, 2006). Cependant, elle comporte également un aspect pratique, à travers l'étude de cas concrets : les plateformes Jumia et Glovo (commerce et livraison en ligne), Yango (transport numérique), ainsi que les services de paiement mobile comme Orange Money et Wave. Cette approche mixte permet de confronter la théorie à la pratique terrain.

\subsection*{Sources de recherche}

\subsubsection*{Sources juridiques}

Les principales sources utilisées seront les textes législatifs et réglementaires ivoiriens, notamment la loi sur la protection des données à caractère personnel, la loi sur les transactions électroniques, ainsi que les textes de la CEDEAO et de l'UEMOA relatifs à l'économie numérique. Seront également pris en compte les conventions internationales et instruments africains encadrant le numérique, tels que la Convention de Malabo sur la cybersécurité et la protection des données.

\subsubsection*{Sources doctrinales et institutionnelles}

Des ouvrages, articles scientifiques, rapports institutionnels (ARTCI, Ministère de la Communication et de l'Économie numérique, Banque centrale, etc.) viendront compléter l'analyse. L'objectif est de confronter les textes à la réalité des pratiques observées sur le terrain ivoirien, en intégrant les perspectives des acteurs publics et privés.

\subsubsection*{Sources secondaires}

Les ressources du web, les catalogues de bibliothèques universitaires nationaux et internationaux, les bases de données bibliographiques pluridisciplinaires et spécialisées, ainsi que les moteurs de recherche généralistes et académiques constitueront des outils complémentaires. Parmi les plateformes consultées figurent notamment SUDOC, WorldCat, DOAJ, ScienceDirect, Web of Science, Scopus et Google Scholar.

\subsection*{Démarche adoptée}

La démarche méthodologique adoptée suit une logique progressive et structurée :

\begin{itemize}
\item Identifier les lacunes et insuffisances du cadre juridique actuel à travers une revue documentaire exhaustive.
\item Examiner les mécanismes de régulation existants ou envisagés par l'État ivoirien, en analysant les politiques publiques et les actions institutionnelles.
\item Comparer ces dispositifs avec ceux d'autres pays africains (par exemple le Sénégal ou le Maroc) pour identifier les bonnes pratiques transférables.
\item Et enfin, proposer des pistes d'amélioration pour une régulation plus efficace et adaptée aux réalités locales, tenant compte du contexte socio-économique ivoirien.
\end{itemize}

\subsection*{Limites de la recherche}

Cette étude pourrait se heurter à certaines limites inhérentes à la nature du sujet :

\begin{itemize}
\item Le manque de données récentes ou d'études locales approfondies sur les marchés numériques en Côte d'Ivoire, en raison de la nouveauté du phénomène.
\item La rareté de la jurisprudence ivoirienne dans ce domaine encore émergent, limitant les analyses jurisprudentielles.
\item Et la difficulté d'accès à certaines informations institutionnelles, souvent peu diffusées ou mises à jour, ce qui pourrait affecter la exhaustivité de l'analyse.
\end{itemize}

\section*{Plan provisoire}

L'étude sera structurée autour d'une introduction générale présentant le contexte et la justification du sujet, suivie de la problématique, des hypothèses de recherche, des objectifs de l'étude et de la méthodologie.

La première partie, "Le cadre et les fondements de la régulation des marchés numériques en Côte d'Ivoire", comprendra deux chapitres : le premier sur l'émergence et la nature des marchés numériques ivoiriens, avec des sections sur les acteurs et secteurs concernés ainsi que les enjeux économiques, sociaux et juridiques ; le deuxième sur le cadre juridique et institutionnel, analysant les textes applicables et le rôle des autorités publiques.

La deuxième partie, "Les défis et perspectives d'une régulation efficace des marchés numériques", abordera dans un premier chapitre les insuffisances de la régulation actuelle (lacunes juridiques et défis socio-économiques), puis dans un deuxième chapitre les pistes d'amélioration pour une gouvernance équilibrée.

Enfin, une conclusion générale dressera le bilan des analyses, formulera des recommandations et ouvrira sur les perspectives futures du numérique en Côte d'Ivoire.